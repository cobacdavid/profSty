\documentclass{article}
\usepackage[a4paysage]{prof}
\pagestyle{empty}
\begin{document}
\begin{multicols}{3}
  \titrepoly{Aide-mémoire}
  \section{\LaTeX}

\begin{Verbatim}
  \repete{4}{$\sum$}
\end{Verbatim}
\begin{minipage}{\linewidth}
  \repete{4}{$\sum$}
\end{minipage}

\medskip
\begin{Verbatim}
  \remplir{5cm}{$\sum$}
\end{Verbatim}
\begin{minipage}{\linewidth}
  \remplir{5cm}{$\sum$}
\end{minipage}

\medskip
\begin{Verbatim}
  \titrepoly{Un titre}
\end{Verbatim}
\begin{minipage}{\linewidth}
  \titrepoly{Un titre}
\end{minipage}

\medskip
\begin{Verbatim}
  \titrepolydeux{Un titre}
\end{Verbatim}
\begin{minipage}{\linewidth}
  \titrepolydeux{Un titre}
\end{minipage}

\medskip
\begin{Verbatim}
  \titrepolysarah{Un titre}
\end{Verbatim}
\begin{minipage}{\linewidth}
  \titrepolysarah{Un titre}
\end{minipage}

\medskip
\begin{Verbatim}
  \impression
\end{Verbatim}
\begin{minipage}{\linewidth}
  \impression
\end{minipage}

\medskip
\begin{Verbatim}
  Une question \pointiles
\end{Verbatim}
\begin{minipage}{\linewidth}
  Une question \pointiles
\end{minipage}

\medskip
\begin{Verbatim}
  Une question \pointilles
\end{Verbatim}
\begin{minipage}{\linewidth}
  Une question \pointilles
\end{minipage}

\medskip
\begin{Verbatim}
  \R \N \Z \D \Q \vi{2} \VI{2} $\vect{AB}$
  \arc{AB} \intff{-1}8 \intoo{-1}8 \intof{-1}8
  \intfo{-1}8 \rg{f} \ora{\text{n'importe quoi}}
  \ora{ABCD} \calt{G} \infa
\end{Verbatim}
\begin{minipage}{\linewidth}
  \R \N \Z \D \Q \vi{2} \VI{2} $\vect{AB}$ \arc{AB} \intff{-1}8
  \intoo{-1}8 \intof{-1}8 \intfo{-1}8 \rg{f} \ora{\text{n'importe quoi}}
  \ora{ABCD} \calt{G} \infa
\end{minipage}

\medskip
\begin{Verbatim}
  \ecrancalc{\ttfamily 1: Type\qquad2:Valeur}
\end{Verbatim}
\begin{minipage}{\linewidth}
  \ecrancalc{\ttfamily 1: Type\qquad2:Valeur}
\end{minipage}

\medskip
\begin{Verbatim}
  \operation{+}{2}{5}
\end{Verbatim}
\begin{minipage}{\linewidth}
  \operation{+}{\operation{-}{4}{8}}{5}
\end{minipage}

\section{\LuaLaTeX}
\medskip
\begin{Verbatim}
  \algoEuclide{1045}{760}
\end{Verbatim}
\begin{minipage}{\linewidth}
  \algoEuclide{1045}{760}  
\end{minipage}

\medskip
\begin{Verbatim}
  \algoSSS{1045}{760}
\end{Verbatim}
\begin{minipage}{\linewidth}
  \algoSSS{1045}{760}  
\end{minipage}

\medskip
\begin{Verbatim}
  \pgcd{1045}{760}
\end{Verbatim}
\begin{minipage}{\linewidth}
  \pgcd{1045}{760}  
\end{minipage}

\medskip
\begin{Verbatim}
  \premiersEntreEux{1045}{760}
\end{Verbatim}
\begin{minipage}{\linewidth}
  \premiersEntreEux{1045}{760}  
\end{minipage}

\medskip
\begin{Verbatim}
  \listeDiviseurs{1045}
\end{Verbatim}
\begin{minipage}{\linewidth}
  \listeDiviseurs{1045}
\end{minipage}

\medskip
\begin{Verbatim}
  \pythRedac{"A"}{"B"}{"C"}
\end{Verbatim}
\begin{minipage}{\linewidth}
  \pythRedac{"A"}{"B"}{"C"}
\end{minipage}

\medskip
\begin{Verbatim}
  \pythCalculHyp{"BC"}{2.8}{4.6}
\end{Verbatim}
\begin{minipage}{\linewidth} 
  \pythCalculHyp{"BC"}{2.8}{4.6} 
\end{minipage}

\medskip
\begin{Verbatim}
  \pythCalculCote{"BC"}{2.8}{4.6}
\end{Verbatim}
\begin{minipage}{\linewidth}
  \pythCalculCote{"BC"}{4.6}{2.8} 
\end{minipage}

\medskip
\begin{Verbatim}
  \begin{luacode}
    p = luatexperso.Pc:new({valeurs ={5}})
    t = {"ajoute(8)", "soustrait(4)",
      "multiplie(2)", "ajouteVI()", "carre()"}
    luatexperso.Pc.applique("p",t)
    p:sortieLaTeX()
  \end{luacode}
\end{Verbatim}
\begin{minipage}{\linewidth}
\begin{luacode}
    p = luatexperso.Pc:new({valeurs ={5}})
    t = {"ajoute(8)", "soustrait(4)", "multiplie(2)", "ajouteVI()", "carre()"}
    luatexperso.Pc.applique("p",t)
    p:sortieLaTeX()
  \end{luacode}
\end{minipage}

\newpage

\section{\MP}

\titrepolysarah{analyse}
\begin{Verbatim}
  exp (expr x)
  ln  (expr x)
  log (expr x)
  trace (suffix f)(expr a,b,inc)
  pointTrace (suffix f)(expr x)
  suite (suffix f)(expr debut,nb)
  tangente (suffix f)(expr x)
  pointTangente (suffix f)(expr x)
\end{Verbatim}

\titrepolysarah{const}
\begin{Verbatim}
  rouge vert bleu jaune
  violet cyan noir orange
\end{Verbatim}
\columnbreak
\titrepolysarah{geo}
\begin{Verbatim}
  barydeux (expr a,alpha,b,beta)
  barytrois (expr a,alpha,b,beta,c,gamma)
  centredegravite (expr a,b,c)
  orthocentre (expr a,b,c)
  centrecerclecirconscrit (expr a,b,c)
  cerclecirconscrit (expr a,b,c)
  centrecercleinscrit (expr a,b,c)
  cercleinscrit (expr a,b,c)
  projete (expr a,b,c)
  rotation@# (expr p,o)
  symetrieaxiale (expr p,a,b)
  symetriecentrale (expr p,o)
  note_angle@# (expr a,o,b,e)
  noteangle (expr a,o,b,e)
  cheminfleche (expr a,b,e)
  noteangledroit (expr a,o,b,e)
  droite@# (expr a,b)
  cube (expr depart,dimarete)
  nommecube
  tetraedre (expr depart,unite)
  tetraedreregulier (expr depart,unite)
  nommetetraedre
  marksize=4pt
  draw_mark (expr p,a)
  draw_marked (expr p,n)
  patate
  hachurage(expr chemin, angle, ecart, trace)suffix couleur
  meshachures(expr chemin, angle, ecart)suffix motif
  vonkoch (expr i,j)
\end{Verbatim}

\titrepolysarah{graphe}
\begin{Verbatim}
  liencourbe (suffix a,b)
  ponderation@# (suffix a,b)(text t)
  pondseul@#(expr p)(text t)
  lien (suffix a,b)
  lienoriente (suffix a,b)
  sommet (suffix a) (expr couple,t)
  sommetcouleur (suffix a) (expr couple) (expr couleur)
\end{Verbatim}

\titrepolysarah{repere}
\begin{Verbatim}
  papiermi@# (expr larg,long)
  quadrillage@# (expr larg,long,espace)
  base@# (expr unitex,unitey)
  quadrillagesimple (expr larg,long,espace)
  axeh@# (expr xmin, xmax, inc, u)
  axev@# (expr ymin, ymax, inc, u)
\end{Verbatim}

\titrepolysarah{stat}
\begin{Verbatim}
  bam (expr vmin,vmax,un,med,trois,larg,haut,pos)
  
\end{Verbatim}
\end{multicols}

\end{document}  

%%% Local Variables: 
%%% mode: latex
%%% TeX-master: t
%%% End: 
