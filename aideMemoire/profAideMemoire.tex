\documentclass{article}

\usepackage[a4]{prof}
\pagestyle{empty}


\begin{document}

\titrepoly{Aide-mémoire}{mémo}

%% \begin{multicols}{1}
  \section{\LaTeX}
  \subsection{Couleurs} 

  \begin{Verbatim}
    \color{\pybleu}\rule{\linewidth}{12pt}
    \color{\pyjaune}\rule{\linewidth}{12pt}
    \color{\macouleur}\rule{\linewidth}{12pt}
    \color{\gris}\rule{\linewidth}{12pt}
    \color{\blanc}\rule{\linewidth}{12pt}
    \color{\noir}\rule{\linewidth}{12pt}
  \end{Verbatim}


  \noindent\color{\pybleu}\rule{\linewidth}{12pt}
  \color{\pyjaune}\rule{\linewidth}{12pt}
  \color{\macouleur}\rule{\linewidth}{12pt}
  \color{\gris}\rule{\linewidth}{12pt}
  \color{\blanc}\rule{\linewidth}{12pt}
  \color{\noir}\rule{\linewidth}{12pt}

  \subsection{Titres}

  \begin{Verbatim}
    \titrepoly{Un titre}{ss-titre}
  \end{Verbatim}
  \begin{minipage}{\linewidth}
    \titrepoly{Un titre}{ss-titre}
  \end{minipage}
  
  \begin{Verbatim}
    \nsiEntete{Un titre}{ss-titre}
  \end{Verbatim}
  \begin{minipage}{\linewidth}
    \nsiEntete{Un titre}{ss-titre}
  \end{minipage}

  \subsection{Répétitions/Remplissages}

  \begin{Verbatim}
    \repete{4}{$\sum$}
  \end{Verbatim}
  \begin{minipage}{\linewidth}
    \repete{4}{$\sum$}
  \end{minipage}
  
  \medskip
  \begin{Verbatim}
    \remplir{5cm}{$\sum$}
  \end{Verbatim}
  \begin{minipage}{\linewidth}
    \remplir{5cm}{$\sum$}
  \end{minipage}

  \subsection{Maths}

  \begin{Verbatim}
    \R \N \Z \D \Q \vi{2}$\vect{AB}$
    \intff{-1}8 \intoo{-1}8
    \intof{-1}8 \intfo{-1}8
    \rg{f} \ora{\text{n'importe quoi}}
    \ora{ABCD} \calt{G} \infa \supa
    \norme{AB} \vectu \vectv \cvect{-3}{2}
    \ron $\sqrt{100}$ $\oldsqrt[3]{100}$
  \end{Verbatim}
  \begin{minipage}{\linewidth}
    \R \N \Z \D \Q \vi{2}$\vect{AB}$
    \intff{-1}8 \intoo{-1}8
    \intof{-1}8 \intfo{-1}8
    \rg{f} \ora{\text{n'importe quoi}}
    \ora{ABCD} \calt{G} \infa \supa
    \norme{AB} \vectu \vectv \cvect{-3}{2}
    \ron $\sqrt{100}$ $\oldsqrt[3]{100}$
  \end{minipage}
  
  \begin{Verbatim}
\fonc{f}{\intff{-5}{5}}{t}{\cos\left(t +\right)}
  \end{Verbatim}
  \begin{minipage}{1.0\linewidth}
    \fonc{f}{\intff{-5}{5}}{t}{\cos\left( t \right)}
  \end{minipage}

  \begin{minipage}{1.0\linewidth}
    \begin{definition}
      env. definition
    \end{definition}

    \begin{exemple}
      env. exemple
    \end{exemple}

    \begin{propriete}
      env propriete
    \end{propriete}

    \begin{theoreme}
      env. theoreme
    \end{theoreme}
    
    \begin{demo}
      env. demo
    \end{demo}

    \begin{remarque}
      env. remarque
    \end{remarque}

    \begin{exercice}
      env. exercice
    \end{exercice}

    \begin{Verbatim}
      \nom
    \end{Verbatim}
    \begin{minipage}{1.0\linewidth}
      \nom
    \end{minipage}

    \begin{Verbatim}
      \devoir{Surveillé}{3}{Mai 2020}{Seconde}
    \end{Verbatim}
    \begin{minipage}{1.0\linewidth}
      \devoir{Surveillé}{3}{Mai 2020}{Seconde}
    \end{minipage}
  \end{minipage}

  \begin{Verbatim}
    $\begin{tabvar}{|C|CCCCCCC|}
      \hline
      x & -\infty & & -3 & & 4 & & +\infty \\\hline
      x+3  & & - & \barre{0} & + & \barre{}  & + & \\\hline
      -x+4 & & + & \barre{}  & + & \barre{0} & - & \\\hline
      \sgnduq& & - & \barre{0} & + & \dbarre{} & - & \\\hline
    \end{tabvar}$
  \end{Verbatim}
  \begin{minipage}{1.0\linewidth}
    $\begin{tabvar}{|C|CCCCCCC|}
      \hline
      x & -\infty & & -3 & & 4 & & +\infty \\\hline
      x+3  & & - & \barre{0} & + & \barre{}  & + & \\\hline
      -x+4 & & + & \barre{}  & + & \barre{0} & - & \\\hline
      \sgnduq& & - & \barre{0} & + & \dbarre{} & - & \\\hline
    \end{tabvar}$ 
  \end{minipage}


  \begin{Verbatim}
    $\begin{tabvar}{|C|CCCCC|} \hline
      x & -\infty & & \dfrac56 & & +\infty \\\hline
      f'(x) & & - & \barre{0} & + & \\\hline
      \niveau{2}{2}
      \varde{f} & & \decroit & f\left(\dfrac56\right) & \croit &  \\\hline
    \end{tabvar}$
  \end{Verbatim}
  \begin{minipage}{1.0\linewidth}
    $\begin{tabvar}{|C|CCCCC|} \hline
      x & -\infty & & \dfrac56 & & +\infty \\\hline
      f'(x) & & - & \barre{0} & + & \\\hline
      \niveau{2}{2}
      \varde{f} & & \decroit & f\left(\dfrac56\right) & \croit &  \\\hline
    \end{tabvar}$
  \end{minipage}
  
  \subsection{Codes}

  \subsubsection{Général}

  \begin{Verbatim}
    \geogebra \lua
    \titrealgo{Tri d'une liste}
  \end{Verbatim}
  \begin{minipage}{1.0\linewidth}
    \geogebra \lua
    \titrealgo{Tri d'une liste}
  \end{minipage}

  \begin{Verbatim}
\begin{verbatimgg}
  Séquence(Point(0,k), k, 0, 100)
\end{verbatimgg}
  \end{Verbatim}
  \begin{minipage}{1.0\linewidth}
    \begin{verbatimgg}
Séquence(Point(0,k), k, 0, 100) 
    \end{verbatimgg}
  \end{minipage}

  \begin{Verbatim}
\begin{verbatimlua}
  f = function (n1, n2) return n1+n2; end
\end{verbatimlua}  
  \end{Verbatim}
  \begin{minipage}{1.0\linewidth}
    \begin{verbatimlua}
      f = function (n1, n2) return n1+n2; end
    \end{verbatimlua}
  \end{minipage}
  \subsection{Divers}
  
  \begin{Verbatim}
    \impression
  \end{Verbatim}
  \begin{minipage}{\linewidth}
    \impression
  \end{minipage}

  \begin{Verbatim}
    Une question \pointiles
  \end{Verbatim}
  \begin{minipage}{\linewidth}
    Une question \pointiles
  \end{minipage}
  
  \medskip
  \begin{Verbatim}
    Une question \pointilles
  \end{Verbatim}
  \begin{minipage}{\linewidth}
    Une question \pointilles
  \end{minipage
}
\newpage
  \subsubsection{Langages}

  \begin{minipage}{1.0\linewidth}
    Environnement \texttt{codepython} ou commande \verb+\fichierpython+
    \fichierpython{exemple_python.py}
  \end{minipage}

  \begin{minipage}{1.0\linewidth}
    Environnement \texttt{codehtml} ou commande \verb+\fichierhtml+
    \fichierhtml{exemple_html.html}
  \end{minipage}

  \begin{minipage}{1.0\linewidth}
    Environnement \texttt{codecss} ou commande \verb+\fichiercss+
    \fichiercss{exemple_css.css}
  \end{minipage}

  \begin{minipage}{1.0\linewidth}
    Environnement \texttt{codejavascript} ou commande \verb+\fichierjavascript+
    \fichierjavascript{exemple_javascript.js}
  \end{minipage}
  
\section{pythontex}

  \subsection{emacs}
  \texttt{Control-c Control-t}

  Parfois la compilation échoue (doc. non actualisé) mais fonctionne directement en
  ligne de commande...
  
  \subsection{pythontex basique}
  
  \begin{enumerate}
  \item \verb|\py|~: évaluation et écrit ce qui est renvoyé en \texttt{string}

    \verb|\py{str(2**3)}| donne \py{str(2**3)}
    
  \item \verb|\pyc| et env. \verb|\pycode|~: évaluation et écrit les \texttt{print}
    
    \begin{Verbatim}
      \pyc{print("OK",end="")}
      \begin{pycode}
      print(r"\begin{tabular}{c|c}")
      print(r"$m$ & $2^m$ \\ \hline")
      print(r"%d & %d \\" % (1, 2**1))
      print(r"%d & %d \\" % (2, 2**2))
      print(r"%d & %d \\" % (3, 2**3))
      print(r"%d & %d \\" % (4, 2**4))
      print(r"\end{tabular}")
      \end{pycode}
    \end{Verbatim}
    \pyc{print("OK")}
    
    \begin{pycode}
print(r"\begin{tabular}{c|c}")
print(r"$m$ & $2^m$ \\ \hline")
print(r"%d & %d \\" % (1, 2**1))
print(r"%d & %d \\" % (2, 2**2))
print(r"%d & %d \\" % (3, 2**3))
print(r"%d & %d \\" % (4, 2**4))
print(r"\end{tabular}") 
\end{pycode}

\item \verb|\pycon| et env. \verb|pyconsole| renvoie l'exécution
  dans une console (sans l'entrée pour la commande)

  \pycon{[i**2 for i in range(10)]}
\begin{pyconsole}
a = [i**2 for i in range(10)]
sum(a)
\end{pyconsole}
\end{enumerate}

\subsection{pythontex commandes persos}

\subsubsection{dft utilisable avec py}
Import de \texttt{profPythonPY} qui contient les biblios~:
\texttt{random}, \texttt{math}, \texttt{requests}, \texttt{imgkit},
\texttt{shutil} et \texttt{dvtdecimal} (as dvt)

\verb|\py[dft]{prof_dec_let("mathématiques")}| donne~:\\
\py[dft]{prof_dec_let("mathématiques")}

\medskip
\verb|\py[dft]{prof_mel_mot("mathématiques")}| donne~:\\
\py[dft]{prof_mel_mot("mathématiques")}

\medskip
\verb|\py[dft]{prof_mel_phr("la vie est belle")}| donne~:\\
\py[dft]{prof_mel_phr("la vie est belle")}

\medskip
\verb|\py[dft]{prof_tab_val(lambda x:x**2, -3, 2, .5, "t", "c(t)", precision=2)}| donne~:\\
\py[dft]{prof_tab_val(lambda x:x**2, -3, 2, .5, "t", "c(t)", precision=2)}

\medskip
\verb|\py[dft]{prof_tab_ind(9, "positif", "liste", "négatif")}| donne~:\\
\py[dft]{prof_tab_ind(9, "positif", "liste", "négatif")}

\medskip
\verb|\py[dft]{prof_oeis_A(45, nb_termes=20)}| donne~:\\
\py[dft]{prof_oeis_A(45, nb_termes=20)}

\medskip
\verb|\py[dft]{prof_oeis_web_A(306475, angle=0)}| donne~:\\
\py[dft]{prof_oeis_web_A(306475, angle=0)}

\medskip
\verb|\py[dft]{prof_tri_bulle([1, 3, 0, 9, 5, 7, 4, 6, 2, 8], avec_couleur=True)}| donne~:\\
\py[dft]{prof_tri_bulle([1, 3, 0, 9, 5, 7, 4, 6, 2, 8], avec_couleur=True)}

\medskip
\verb|\py[dft]{prof_tri_insertion([1, 3, 0, 9, 5, 7, 4, 6, 2, 8], avec_couleur=True)}| donne~:\\
\py[dft]{prof_tri_insertion([1, 3, 0, 9, 5, 7, 4, 6, 2, 8], avec_couleur=True)}

\medskip
\verb|\py[dft]{prof_tri_selection([1, 3, 0, 9, 5, 7, 4, 6, 2, 8], avec_couleur=True)}| donne~:\\
\py[dft]{prof_tri_selection([1, 3, 0, 9, 5, 7, 4, 6, 2, 8], avec_couleur=True)}

\medskip
\verb|\py[dft]{prof_image_site("https://www.oeis.org", "oeis.png", texte="oeis", dimension=r".9\linewidth")}| donne~:\\
\py[dft]{prof_image_site("https://www.oeis.org", "oeis.png", texte="site de oeis", dimension=r".9\linewidth")}

\medskip
\verb|\py[dft]{prof_BM_mc("nsi et isn ne sont pas identiques.", "ne sont pas")}| donne~:\\
\py[dft]{prof_BM_mc("nsi et isn ne sont pas identiques.", "ne sont pas")}

\medskip
\verb|\py[dft]{prof_BM_mc("51528995084834006858415458848660202", "848")}| donne~:\\
\py[dft]{prof_BM_mc("51528995084834006858415458848660202", "848")}

\medskip
\verb|\py[dft]{prof_arbre_binaire(noeud(1, noeud(2, noeud(4, noeud(6)), noeud(5)), noeud(3)))}| donne~:\\
\py[dft]{prof_arbre_binaire(noeud(1,
  noeud(2, noeud(4, noeud(6)), noeud(5)),
  noeud(3))
  )}

\medskip
\verb|\py[dft]{prof_arbre_binaire_complet(3, "->", "scale=.5")}| donne~:\\
\py[dft]{prof_arbre_binaire_complet(3, "->", "scale=.5")}


\subsubsection{math}

Import pour calcul comme dans mon plugin \texttt{calc} de \texttt{zsh}

\verb|La dérivée de $f(x)=\cos(x)$ est $f\prime(x)=\py[math]{latex(diff(cos(x), x))}$| donne~:\\
La dérivée de $f(x)=\cos(x)$ est $f\prime(x)=\py[math]{latex(diff(cos(x), x))}$

%% \end{multicols}

% \medskip
% \begin{Verbatim}
%   \operation{+}{2}{5}
% \end{Verbatim}
% \begin{minipage}{\linewidth}
%   \operation{+}{\operation{-}{4}{8}}{5}
% \end{minipage}

% \section{\LuaLaTeX}
% \medskip
% \begin{Verbatim}
%   \algoEuclide{1045}{760}
% \end{Verbatim}
% \begin{minipage}{\linewidth}
%   \algoEuclide{1045}{760}  
% \end{minipage}

% \medskip
% \begin{Verbatim}
%   \algoSSS{1045}{760}
% \end{Verbatim}
% \begin{minipage}{\linewidth}
%   \algoSSS{1045}{760}  
% \end{minipage}

% \medskip
% \begin{Verbatim}
%   \pgcd{1045}{760}
% \end{Verbatim}
% \begin{minipage}{\linewidth}
%   \pgcd{1045}{760}  
% \end{minipage}

% \medskip
% \begin{Verbatim}
%   \premiersEntreEux{1045}{760}
% \end{Verbatim}
% \begin{minipage}{\linewidth}
%   \premiersEntreEux{1045}{760}  
% \end{minipage}

% \medskip
% \begin{Verbatim}
%   \listeDiviseurs{1045}
% \end{Verbatim}
% \begin{minipage}{\linewidth}
%   \listeDiviseurs{1045}
% \end{minipage}

% \medskip
% \begin{Verbatim}
%   \pythRedac{"A"}{"B"}{"C"}
% \end{Verbatim}
% \begin{minipage}{\linewidth}
%   \pythRedac{"A"}{"B"}{"C"}
% \end{minipage}

% \medskip
% \begin{Verbatim}
%   \pythCalculHyp{"BC"}{2.8}{4.6}
% \end{Verbatim}
% \begin{minipage}{\linewidth} 
%   \pythCalculHyp{"BC"}{2.8}{4.6} 
% \end{minipage}

% \medskip
% \begin{Verbatim}
%   \pythCalculCote{"BC"}{2.8}{4.6}
% \end{Verbatim}
% \begin{minipage}{\linewidth}
%   \pythCalculCote{"BC"}{4.6}{2.8} 
% \end{minipage}

% \medskip
% \begin{Verbatim}
%   \begin{luacode}
%     p = luatexperso.Pc:new({valeurs ={5}})
%     t = {"ajoute(8)", "soustrait(4)",
%       "multiplie(2)", "ajouteVI()", "carre()"}
%     luatexperso.Pc.applique("p",t)
%     p:sortieLaTeX()
%   \end{luacode}
% \end{Verbatim}
% \begin{minipage}{\linewidth}
% \begin{luacode}
%     p = luatexperso.Pc:new({valeurs ={5}})
%     t = {"ajoute(8)", "soustrait(4)", "multiplie(2)", "ajouteVI()", "carre()"}
%     luatexperso.Pc.applique("p",t)
%     p:sortieLaTeX()
%   \end{luacode}
% \end{minipage}

% \end{multicols}
% \newpage

% \section{\MP}

% \titrepolysarah{analyse}
% \begin{Verbatim}
%   exp (expr x)
%   ln  (expr x)
%   log (expr x)
%   trace (suffix f)(expr a,b,inc)
%   pointTrace (suffix f)(expr x)
%   suite (suffix f)(expr debut,nb)
%   tangente (suffix f)(expr x)
%   pointTangente (suffix f)(expr x)
% \end{Verbatim}

% \titrepolysarah{const}
% \begin{Verbatim}
%   rouge vert bleu jaune
%   violet cyan noir orange
% \end{Verbatim}
% \columnbreak
% \titrepolysarah{geo}
% \begin{Verbatim}
%   barydeux (expr a,alpha,b,beta)
%   barytrois (expr a,alpha,b,beta,c,gamma)
%   centredegravite (expr a,b,c)
%   orthocentre (expr a,b,c)
%   centrecerclecirconscrit (expr a,b,c)
%   cerclecirconscrit (expr a,b,c)
%   centrecercleinscrit (expr a,b,c)
%   cercleinscrit (expr a,b,c)
%   projete (expr a,b,c)
%   rotation@# (expr p,o)
%   symetrieaxiale (expr p,a,b)
%   symetriecentrale (expr p,o)
%   note_angle@# (expr a,o,b,e)
%   noteangle (expr a,o,b,e)
%   cheminfleche (expr a,b,e)
%   noteangledroit (expr a,o,b,e)
%   droite@# (expr a,b)
%   cube (expr depart,dimarete)
%   nommecube
%   tetraedre (expr depart,unite)
%   tetraedreregulier (expr depart,unite)
%   nommetetraedre
%   marksize=4pt
%   draw_mark (expr p,a)
%   draw_marked (expr p,n)
%   patate
%   hachurage(expr chemin, angle, ecart, trace)suffix couleur
%   meshachures(expr chemin, angle, ecart)suffix motif
%   vonkoch (expr i,j)
% \end{Verbatim}

% \titrepolysarah{graphe}
% \begin{Verbatim}
%   liencourbe (suffix a,b)
%   ponderation@# (suffix a,b)(text t)
%   pondseul@#(expr p)(text t)
%   lien (suffix a,b)
%   lienoriente (suffix a,b)
%   sommet (suffix a) (expr couple,t)
%   sommetcouleur (suffix a) (expr couple) (expr couleur)
% \end{Verbatim}

% \titrepolysarah{repere}
% \begin{Verbatim}
%   papiermi@# (expr larg,long)
%   quadrillage@# (expr larg,long,espace)
%   base@# (expr unitex,unitey)
%   quadrillagesimple (expr larg,long,espace)
%   axeh@# (expr xmin, xmax, inc, u)
%   axev@# (expr ymin, ymax, inc, u)
% \end{Verbatim}

% \titrepolysarah{stat}
% \begin{Verbatim}
%   bam (expr vmin,vmax,un,med,trois,larg,haut,pos)
  
% \end{Verbatim}
% \end{multicols}

\end{document}  

%%% Local Variables: 
%%% mode: latex
%%% TeX-master: t
%%% End: 
